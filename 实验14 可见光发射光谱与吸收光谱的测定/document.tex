\documentclass[a4paper,12pt]{article}
\usepackage[utf8]{inputenc}
\usepackage{amsmath, amssymb}
\usepackage{graphicx}
\usepackage{booktabs}
\usepackage{array}
\usepackage{float}
\usepackage{geometry}
\usepackage{makecell}
\usepackage{ctex}
\geometry{a4paper, margin=1in}

\begin{document}
	
	\section*{\textbf{四、实验数据和实验现象记录表}}
	
	\subsection*{4.1 发射光谱测试记录表}
	\begin{table}[htbp]
		\centering
		\begin{tabular}{|p{3cm}|p{3cm}|p{4cm}|p{5cm}|}
			\hline
			\textbf{光源} & \textbf{测试日期} & \textbf{主要峰/带位置 (nm)} & \textbf{光谱特征及现象} \\
			\hline
			荧光灯 &  & \makecell[l]{~435、545、611\\等} & 离散多峰,含汞发射\\ + 荧光粉辅助发光 \\
			\hline
			氙灯 &  & ~350 ~ 800 范围 & 连续带 + 氙特征发射线 \\
			\hline
			卤钨灯 &  & ~400 ~ 800 连续 & 类似黑体辐射的连续光谱 \\
			\hline
			白光 LED &  & 450 nm + 宽带发射 & \makecell[l]{450 nm 蓝光芯片\\ + 荧光粉形成白光} \\
			\hline
			太阳 (室外) &  & ~400 ~ 800 连续 & \makecell[l]{连续光谱,弱吸收特征\\ (若设备灵敏度足够)} \\
			\hline
		\end{tabular}
		\caption{发射光谱测试记录表}
	\end{table}
	
	\subsection*{4.2 吸收光谱测试记录表}
	(a) 三氯化六氨合钴
	
	\begin{table}[htbp]
		\centering
		\begin{tabular}{|p{3cm}|p{3cm}|p{3cm}|p{3cm}|p{4cm}|}
			\hline
			\textbf{溶液名称} & \textbf{浓度 (g/20 mL)} & \textbf{主要吸收峰 (nm)} & \textbf{吸收峰对应能量 (eV)} & \textbf{现象描述} \\
			\hline
			$[\text{Co(NH}_3\text{)}_6]$Cl$_3$ & 0.1 / 20 mL & \makecell[l]{例:470、550} & \makecell[l]{例:2.64、2.25} & 溶液呈淡红~橙色,\\有特征吸收带 \\
			\hline
		\end{tabular}
		\caption{三氯化六氨合钴吸收光谱测试记录表}
	\end{table}
	
	(b) CuInS$_2$ 量子点
	
	\begin{table}[htbp]
		\centering
		\begin{tabular}{|p{3cm}|p{3cm}|p{3cm}|p{3cm}|p{3cm}|p{4cm}|}
			\hline
			\textbf{样品名称} & \textbf{参比} & \textbf{起峰点 (nm)} & \textbf{起峰点对应能量 (eV)} & \textbf{与 1.50 eV 比较} & \textbf{现象描述} \\
			\hline
			CuInS$_2$ (QD) & 正己烷 & \makecell[l]{例:600 nm} & \makecell[l]{例:2.07 eV} & 大于 1.50 eV → 量子点 & \makecell[l]{溶液呈浅色~橙黄透明,\\带荧光} \\
			\hline
		\end{tabular}
		\caption{CuInS$_2$ 量子点吸收光谱测试记录表}
	\end{table}
	
	\subsection*{4.3 荧光光谱测试记录表}
	\begin{table}[htbp]
		\centering
		\begin{tabular}{|p{3cm}|p{3cm}|p{4cm}|p{3cm}|p{4cm}|}
			\hline
			\textbf{样品名称} & \textbf{激发光源 (nm)} & \textbf{荧光峰范围 (nm)} & \textbf{发射峰峰顶 (nm)} & \textbf{现象描述} \\
			\hline
			CuInS$_2$ (QD) & 450 nm & \makecell[l]{例:550 - 800 nm\\宽带} & \makecell[l]{例:650 nm} & \makecell[l]{显示可见光区内\\明显的宽带发射} \\
			\hline
		\end{tabular}
		\caption{荧光光谱测试记录表}
	\end{table}
	
\end{document}
